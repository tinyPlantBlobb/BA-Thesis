%% Karlsruhe Institute of Technology
%% Institute for Anthropomatics and Robotics (IAR)
%% Artificial Intelligence for Language Technologies (AI4LT) lab
%%
%% Prof. Dr. Jan Niehues
%% Lab's website https://ai4lt.anthropomatik.kit.edu/english/index.php


%TODO add methedology and move from experiments part

\chapter{Experiments}
\label{ch:experiment}
This chapter details the dataset and the experiments that have been run on Whisper\cite{radford2022robust}, Seamless\cite{seamless2023} and DeltaLM\cite{ma2021deltalm}.
The experiments on whisper and seamless have been made with the help of the huggingface \cite{huggingface} models and frame works, where as DeltaLm has been used with the fairseq toolkit \cite{ott2019fairseqfastextensibletoolkit}.

\chapter{the Dataset}
\label{ch:Dataset}
The used dataset for all the scores is the benchmark section of the IWSLT2023 \cite{sperber2024evaluating}, which consists of TED talks and contains both reference translations and reference transcriptions, as well as a segmentation for the transcriptions, which matches roughly the translation segmentation. 
\section{Segmentation}
\label{sec:FirstContent:Segmentation}
Since most ASR models and end-to-end speech translation models only take audio up to a certain length as input the given audio files have to be segemented into smaller chunks. 
The segmentation of the audio from the data set will be done with the timestamps given in the dataset itself. The timestamps given in the dataset correspond to the source language transcription reference and the target language translation also correspond mostly well with the reference segments given in the dataset. 
For the segments that do not overlap properly or are shifted by a segment in the given xml, mwerSegementer is used to align the outputs with the references properly, so the comet scores can be calculated on those alrigned sentences and the computed reference scores fit. 
mwerSegmenter minimises the WER between 2 segments,  for this the model transcription or translation is used as the gold segmentation to which the reference is aligned 
%TODO move stuff to the right places
\section{Translation and transcriptions}
Since all models that are worked with are Encoder-decoder-models that use the transformer architecture the resulting formulas for the resulting probabilities are similar, for completeness the probability formula per model is listed separately. 

\subsection{Transcription Probability}
the Transcription Probability is calculated in a similar manner as the translation probability, which results in the formula $$-\frac{1}{T}\sum_{t=1}^T log p(y_t^v)$$ where T is the number of decoding steps and p is the probability.
Since only the token with the highest probability is looked at those proabbilities are always bigger than 0 so there are no potential issues with 0 values. (as if the probabilities of all regualr tokens is 0 the end of section token would have a non 0 probability)

The Transcription step is only done on the Whisper model.
Whisper \cite{radford2022robust} is a multitask and multilanguage model for Automatic speech recognition as well as speech translation. It is primarily used as an ASR model in this thesis. 
Open AI gives several different sizes for whisper models, in this thesis the medium model is used, specifically the pretrained model that is available on huggingface, which provides a processor and a few different whisper models that have different head on top of it, the basic Whisper model on huggingface outputs the raw hidden states without a specific head on-top of it. 
The specific model version used in the experiments is the WhisperForConditionalGeneration, as it has a language modelling head, and is recommended for automatic speech recognition, there are also ones that have heads for audio classification or a language modelling head that is a linear layer with weights tied to the input embeddings. 

To retrieve the final transcription probability the huggingface model returns a tuple (or dictionary) that contains the probability of each vocabulary entry at each decoding step.
Those values are then processed so that the sum of the highest probabilities is taken and then normalised by the number of decoding steps that were done, this gives the transcription probability mean. For reference the pure probability sum was also collected and returned during the experiments. 
THere is a small difference in the resulting QE scores between the two versions but that is more noticable in the dropout version of the transcription probabilty 

\subsection{Translation Probability}
The translation probability is calculated by the formula \ref{formula:translation probability} this is also the case in the seamless and deltalm model. 

\todo{maybe remove subsections?}
\subsection{Seamless Translation probability}
The translation probability follows the general translation probability formula \ref{formula:translation probability}. To get the probability of the top token at each decoding step the functionality from huggingface is used, which is able to return the log-probabilities and scores, so the processed log-proabbilites, of all vocabulary entries. 
Then to get the final probability of the sequence the top token of each decoding step and it's probability is picked and the then summed up and divided by the number of decoding steps that are in the sequence.
The translation probability on seamless is, just like the transcription probability on Whisper, retrieved with the help of the pytorch and the seamless large model on huggingface. 


\subsection{DeltaLM translation probability}
the experiments \cite{ma2021deltalm} were done on a finetuned large version of it that was finetuned using the training data from the IWSLT 2023 constrained category, for this only the english german part was used.
\\
the Text was preprocessed with the pretrained senctencepiece model and dictionary that has been provided on the DeltaLM github page \cite{deltalmurl} after that it is preprocessed with faireq preprocess. the result of this is then put into fairseq generate with batch size and beam size 1 for the softmax entropy 
 dthe running of the experiments was done with the help of the fairseq toolkit \cite{ott2019fairseqfastextensibletoolkit} which gives the probability of the specific translation along with the translation hypothesis and prints the softmax probabilities after each decoding step. by default these probabilites are in the base 2 logarithm. 
\subsection{end-to-end translation probability}
In end-to-end translation there is no intermediate step between the audio and text part so the resulting probability is a single score for the whole process.
the resulting probability formula comes from the architecture, which in this case is a encoder decoder architecture which results in the formula: $$-\frac{1}{T}\sum_{t=1}^T log\; p(y_t)$$ which is the same as the translation probability and the transcription probability. 

\section{Softmax entropy}
in the formicheva \cite{fomicheva2020unsupervised} paper the Softmax entropy meassure was proposed as $$-\frac{1}{T}\sum_{t=1}^T\sum_{v=1}^Vp(y_t^v)logp(y_t^v)$$ with V being the Vocabulary size and T being the length of the translation. Which in this definition doesn't work with the data that the models produce as there are quite a lot of 0 values for the vocabulary after using the softmax, which means that the result of the sum like that would not be defined, as the log of 0 is $-\infty$. To mitigate this the 0 values are masked for the log and by extend the sum, so those values are essentially ignored. 

To get the vocabulary values at each decoding stem from the seamless model basic code has been written that iterates over all vocabulary tokens, which are returned by the model generation by the huggingface model and calculated the entropy of each decoding step
To get them from DeltaLm the fairseq source code was adapted to do the same as the huggingface model, as the fairseq toolkit is unable to do so natively so far. 
In both cases the batch and beam size are set to 1 as otherwise the resulting tensors make it more difficult to pick out which is the right entropy for the resulting batch. 

%softmax entropy from the transcription part? maybe in the future 
\section{standart deviation}
the standard deviation is calculated over the top token probability at each decoding step 
->which doesn't quite match the word probabilities but is close enough 


\section{Dropout}
the dropout based metrics are obtained by doing 30 passes with the same input through the models each time neurons are masked to 0 my some probability, which is usually the same probability as was used in training, or set to 0.1 as this is a common dropout probability. 

\subsection{Whisper Dropout}
for the dropout based uncertainty quantification the model transcribes the same bit of audio 30 times and for each pass the transcription probability and mean probability is computed, in each pass the model randomly masks some nodes to 0
this results in some short and some faulty transcriptions but this also gives information about how certain the model is in it's transcription. the used dropout probability is 0.1 


\subsection{Seamless Dropout}
for the dropout based approach the transcription with the best qe score, so the min/max score from the whisper step, is used 
and then put into seamless which then translates this 30 times with a dropout probabiltity of 0.1.
% potentially remove if finding a better solution
Due to the nature of pytorch and huggingface models the dropout has to be done in training mode, as the evaluation mode turns off any dropout layers that are in the model. Due to this and a bug in the implementation for caching during forward passes in the seamless model on huggingface, which leads to a tuple index out of range error that only appears in train mode with dropout turned on, the caching was turned off caching in the seamless config

\subsection{Deltalm Dropout}
The dropout with the deltaLM model is handled over fairseq, with the --retain-dropout flag which enables the dropout during interference.
The used dropout probability is 0.1 and is applied on all modules that is was applies on during training, this includes the encoder and the decoder, it does not include attention dropout. 
The resulting 


\subsection{End-To-End seamless dropout}
The dropout for the end-to-end approach is once again applies on the decoder and as with the translation part of seamless the caching in the model has been turned off. The dropout probability is 0.1, same as in the translation part. The batch size used for running this is once again 1 and the beamsize is also set to 1. 


%\section{lexsim}
%the lexical similarity between 2 sentences from the dropout experiments are computed with the help of meteor \cite{banerjee-lavie-2005-meteor}
