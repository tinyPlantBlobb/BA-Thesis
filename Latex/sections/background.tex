\chapter{Background}
This chapter explains the methods and concepts these are seperated in 2 parts one for basic knowledge and 

\section{Basic Knowledge}
Important knowledge for the contents of the thesis mainly the what translation, automatic Speech recognition and dropout


\subsection{Automatic Speech Recognition}
Automatic speech recognition systems or short ASR systems are systems that recognise and transcribe spoken language. 
%TODO

\subsection{Translation}
Translation is the practice of translating text or language from one language into another language. This can be done by hand by a human or in a very statical approach where a dictionary is used to directly translate the text 
%TODO
\subsubsection{Speech translation}
Speech translation or spoken language translation is simmilar to the regular translation but it has, like the name says, spoken language as the basis instead of text. 
%TODO


\subsection{Dropout}
- has been utilised in DNN to measure uncertainty
- mainly in Deep architectures but also in Auto-encoders \cite{}
-in ASR it has been tried here  \cite{8683086}
- in ASR it mainly stems from noisy audio 
%TODO


\subsection{Transcription probability}
the probability that the ASR component transcribes the audio to this sequence of text
%TODO


\subsection{Translation probability}
- the probability that the model generates the sequence $y = y_1, y_2 \dots y_n$ for the input $x=x_1, x_2 \dots x_n$
norm the translation probabilty over the length of the transcribed and translation 
%TODO
\subsection{Pearsoncorrelation}
%TODO

\section{Models}
\subsection{Transformer}
%TODO
\subsection{Decoder-Encoder}
%TODO
\subsection{ Cascaded Models}
%TODO
\subsection{End-to-End Models}
%TODO
\subsection{Whisper}
%TODO
\subsection{Seamless}
%TODO

\subsection{DeltaLM}

%TODO