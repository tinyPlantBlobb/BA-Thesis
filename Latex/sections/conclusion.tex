%% Karlsruhe Institute of Technology
%% Institute for Anthropomatics and Robotics (IAR)
%% Artificial Intelligence for Language Technologies (AI4LT) lab
%%
%% Prof. Dr. Jan Niehues
%% Lab's website https://ai4lt.anthropomatik.kit.edu/english/index.php

\chapter{Conclusion}
\label{ch:Conclusion}
The proposed metrics by Fomicheva et. al \cite{fomicheva2020unsupervised} also work in cascaded Speech translation, where the translation parts, like softmax entropy and standard deviation calculation, are done on the translation side of the cascaded model and probability retrieval on the transcription part. 

The dropout based metrics also deliver good correlation results with the reference scores  for both the transcription part of cascaded models as well as the translation part, it also works on end-to-end speech translation.

Two unified scores for cascaded models has been proposed and shown to be a well correlated metric to reference scores, for which a formula also has been proposed. 
An additional change to the unified score has also been proposed where the different scores are weighed differently in the computation however finding a good value for a general value of this would require more testing with several more models and frameworks or tool kits. To explore the impact of there frameworks or toolkits used, as well as the impact of the models used on this score. 

Future works could take a closer look at these metrics for different spoken language pairs as it has been show in the past that these metrics perform even better on low and mid resource languages. 