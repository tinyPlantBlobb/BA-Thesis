%% Karlsruhe Institute of Technology
%% Institute for Anthropomatics and Robotics (IAR)
%% Artificial Intelligence for Language Technologies (AI4LT) lab
%%
%% Prof. Dr. Jan Niehues
%% Lab's website https://ai4lt.anthropomatik.kit.edu/english/index.php

\Abstract
Diese Bachelorarbeit untersucht die unüberwachte Qualitätsschätzung für gesprochene Sprachübersetzung (Spoken Language Translation, SLT) unter Verwendung von Glasbox-Merkmalen aus neuronalen Modellen. 
Mit der zunehmenden Nutzung von maschineller Übersetzung und automatischer Spracherkennung (Automatic Speech Recognition, ASR) wird eine präzise Qualitätseinschätzung entscheidend, um den menschlichen Nachbearbeitungsaufwand zu minimieren. 
Die Studie passt etablierte, unüberwachte Metriken aus textbasierter maschineller Übersetzung an, um SLT-Systeme zu bewerten, und konzentriert sich dabei auf kaskadierte und End-to-End-Modelle. 
Methoden wie Transkriptionswahrscheinlichkeit, Softmax-Entropie und Unsicherheit durch Dropout werden verwendet, um die Qualität von Transkriptionen und Übersetzungen zu schätzen. 
Darüber hinaus wird ein einheitlicher Bewertungsansatz vorgeschlagen, um die Gesamtleistung kaskadierter SLT-Systeme zu bewerten. 
Experimente mit modernen Modellen wie Whisper, SeamlessM4T und DeltaLM zeigen die Wirksamkeit der vorgeschlagenen Metriken in der Korrelation mit standardisierten Evaluationsmethoden. 
Diese Arbeit bietet ein Framework für eine effiziente Qualitätsschätzung, das Verbesserungen in SLT-Systemen ermöglicht, ohne auf umfangreiche gelabelte Daten angewiesen zu sein.