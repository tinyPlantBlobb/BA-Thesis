%% Karlsruhe Institute of Technology
%% Institute for Anthropomatics and Robotics (IAR)
%% Artificial Intelligence for Language Technologies (AI4LT) lab
%%
%% Prof. Dr. Jan Niehues
%% Lab's website https://ai4lt.anthropomatik.kit.edu/english/index.php

\chapter{Introduction}
\label{ch:Introduction}

Translation is a time intense problem, that with the help of machine translation has gotten a lot easier, but since the machine translation is not perfect, humans still need to double check the translations. This process of double checking the translations still takes a lot of time as the human that does the quality check has to read over the entire translation and the source material to be able to gauge if the translation is good. 

Spoken language translation only adds more potential problems to this. if a human would have to translate spoken language, like a video or a speech, they would have two options, firstly transcribing the spoken language into text in the source language and then translating it, or secondly translating it on the go. both of these take a lot of time as depending on the language used finding a good translation can be hard. 
Fortunately Natural Language Processing, or short NLP, has made a lot of advancements in the past years and the technology to make such tasks easier exist already. 
Automatic Speech Recognition, or in short ASR, has gotten quite good and quite fast at transcribing audio in many different languages and Machine translation can translate almost anything. 

But this alone does not help speeding up the process of spoken language translation, as a human would still have to listen to the entire audio to make sure the transcript is right and the translation based off of this also doesn't make mistakes. 
To remedy this Quality Estimation can be employed, which aims to estimate the quality of the transcript and the translation. 


The approach used in this thesis was proposed on text translation by \cite{fomicheva2020unsupervised} and has been adapted to Speech translation. 
%\todo{add more, general stuff, connection to the rest etc.}

