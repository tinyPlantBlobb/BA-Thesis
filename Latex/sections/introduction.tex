%% Karlsruhe Institute of Technology
%% Institute for Anthropomatics and Robotics (IAR)
%% Artificial Intelligence for Language Technologies (AI4LT) lab
%%
%% Prof. Dr. Jan Niehues
%% Lab's website https://ai4lt.anthropomatik.kit.edu/english/index.php

\chapter{Introduction}
\label{ch:Introduction}
Spoken language is everywhere, from talking to people in person, radio transmissions or videos a big part of communication is spoken language. But not every person speaks the same language. Take for example a look at the internet, a very large section of the internet is in English. This includes many websites, hours upon hours of audio in for example the form of podcasts or radio and videos like the many hours of video material uploaded to sites like YouTube. All of this is only accessible if a person speaks English or someone translates it into a different language. 

But translation is a time intense problem, that with the help of machine translation has gotten a lot easier. Since the machine translation is not perfect, humans still need to double check the translations. This process of double checking the translations still takes a lot of time as the human that does the quality check has to read over the entire translation and the source material to be able to gauge if the translation is good. 

Spoken language translation only adds more potential problems to this. If a human would have to translate spoken language, like a video or a speech, they would have two options, firstly transcribing the spoken language into text in the source language and then translating it, or secondly translating it on the go. Both of these take a lot of time as depending on the language used finding a good translation can be hard. 
Fortunately Natural Language Processing, NLP for short, has made a lot of advancements in the past years and the technology to make such tasks easier exist already. 
Automatic Speech Recognition, or in short ASR, has gotten quite good and quite fast at transcribing audio in many different languages and machine translation can translate almost anything. 

But this alone does not help speeding up the process of spoken language translation, as a human would still have to listen to the entire audio to make sure the transcript is right and the translation based off of this transcript also doesn't make mistakes. 
To remedy this Quality Estimation can be employed to estimate the quality of the transcript and the translation. 

Quality Estimation, like the name implies, has the goal of to estimate the quality of a system, a product or a result. 
This thesis takes a closer look at quality estimation of such spoken language translation, more specifically it takes a look at cascaded model spoken language translation and the quality estimation of this. 
To estimate the quality of the spoken language translation, different features from both the ASR and machine translation (MT) systems are used to create a so called white or glass box approach. In addition to that the aimed for quality estimation techniques should be usable in an unsupervised manner, this means that no references are needed. 

The basis for the approach used in this thesis was proposed on text translation by Fomicheva et al. \cite{fomicheva2020unsupervised} and has been adapted to Spoken language translation. 

