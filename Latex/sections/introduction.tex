%% Karlsruhe Institute of Technology
%% Institute for Anthropomatics and Robotics (IAR)
%% Artificial Intelligence for Language Technologies (AI4LT) lab
%%
%% Prof. Dr. Jan Niehues
%% Lab's website https://ai4lt.anthropomatik.kit.edu/english/index.php

\chapter{Introduction}
\label{ch:Introduction}

Translation is a time intense problem, that with the help of machine translation has gotten a lot easier, but since the machine translation is not perfect, humans still need to double check the translations. This process of double checking the translations still takes a lot of time that can be 

The approach used in this thesis was proposed on text translation by \cite{fomicheva2020unsupervised} and has been adapted to Speech translation. 
\section{Explanations}
\subsection{Transcription probability}
the probability that the ASR component transcribes the audio to this sequence of text
\subsection{Translation probability}
- the probability that the model generates the sequence $y = y_1, y_2 \dots y_n$ for the input $x=x_1, x_2 \dots x_n$
norm the translation probabilty over the length of the transcribed and translation 


\subsection{Dropout}
- has been utilised in DNN to measure uncertainty
- mainly in Deep architectures but also in Auto-encoders \cite{}
-in ASR it has been tried here  \cite{8683086}
- in ASR it mainly stems from noisy audio 
